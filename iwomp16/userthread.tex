%\textbf{Definition}
We define a user thread in a program as a thread that is not created by an OpenMP implementation. 
%A user thread 
%could become an OpenMP initial thread. 
The most common examples of user threads are POSIX Threads (PThreads) on POSIX-compliant Unix/Linux systems, 
%There are also implementations of other thread libraries, 
%for example, 
Windows Native threads and language based threading support such as Java threads, C++11 threads and futures, as well as others such as qthreads. 
Most OpenMP implementations do not consider the existence of user threads in the OpenMP runtime scheduling and resource managements. 
The existence of multiple user threads in an OpenMP program adds additional level(s) in the overall ``threading''
hierarchy of a program. % Those additional levels could be on top of OpenMP threading mechanism when 
A user thread could become an OpenMP initial thread that creates OpenMP thread
parallelism, or beneath the OpenMP threading mechanism when an OpenMP thread spawns 
a user thread, or the the mix of both. 

These additional levels of threading increase the complexity of a program, both for 
users in the aspect of reasoning the parallel and synchronization behaviors of a program, 
and also for the implementation in terms of resource management and 
interactions. Adding to the complexity is the fact that a user thread may be created 
through a call to a library function whose paralelism (OpenMP) behavior is not known to 
the callee. Typical issues include the following examples: 
\begin{itemize}
\item Does each user thread use the same OpenMP runtime library or not? 
	If not using the same library, how to handle symbol name 
	conflicts of two more different OpenMP runtime libraries? 
\item For user threads that use the same OpenMP runtime library, does the user threads each create its own runtime instance or they share one?
\item For user threads each of which has its own runtime instance (from the same or 
	different runtime library), how to coordinate the resource management among those
	runtime instances to address such issues as oversubscriptions and the affinity
	between user threads?
\end{itemize}
%\TODO{What if a user thread does not use any OpenMP runtime at all? }

It is important to note that approaches to address those issues are very implementation
dependent, requiring protocol and agreement in the runtime behavior and/or interfaces 
of different OpenMP implementations. It may not be realistic to solve some of the issue
from the language standard, and should be left to users to deal with them. In this
aspect, we still hope this report could provide userful information and practices 
for users. 

\REM{
Figure~\ref{fig:pthread-omp} shows an example of having three user 
threads (two PThreads and one thread of the main program) in an OpenMP program. 
The two three threads execute in parallel after the two PThreads are created. 
Each thread calls a function that will enter into
OpenMP threading parallelism. So they all become OpenMP initial threads. How the 
user threads (PThreads in this example) interact with the OpenMP threading mechanisms
in the runtime is up to the implementation. They may share the same OpenMP runtime
instance or each has its own OpenMP runtime instance. 

In the example from Figure~\ref{fig:pthread-omp}, 
one can view this in a two-level threading parallelism: the top level user thread 
parallelism and the bottom level OpenMP threading parallelism.

\begin{figure}[t]
\centering
  \fbox{
 % \lstset{basicstyle=\ttfamily\scriptsize,language=c}
  \lstset{basicstyle=\ttfamily\scriptsize,language=c,numbers=left, %,frame=single,
  deletekeywords={int,if,else,while},
  morekeywords={pragma,omp,target,device,map,
  tofrom,to,from,alloc,parallel,shared,reduction,data,collapse,
  private,dist_iteration,match_range,halo,exchange},
  numbersep=12pt,numberstyle=\color{red}}
  \lstinputlisting{pthread-omp.c}

}
\caption{Three user threads (two Pthreads and one main thread) with OpenMP}
  \label{fig:pthread-omp}
\end{figure}

\subsection{Impacts and Discussions}
}

