%\textbf{Definition}
A typical example of the Phased class is the use of user threads in a program. A user thread 
is the thread that is explicitly created by user, 
for examples PThreads or Windows Native threads created 
using POSIX or Win32 APIs, or C++11 threads. 
%implementation. 
%A user thread 
%could become an OpenMP initial thread. 
%The most common examples of user threads are POSIX Threads (PThreads) on POSIX-compliant Unix/Linux systems, 
%There are also implementations of other thread libraries, 
%for example, 
%Windows Native threads and language based threading support such as Java threads, C++11 threads and futures, as well as others such as qthreads. 
A user thread could become an OpenMP initial thread that creates OpenMP {\sf parallel} regions, 
or nested inside a {\sf parallel} region when an OpenMP thread spawns 
a user thread, or the mix of both. 

\REM{
Figure~\ref{fig:pthread-omp} shows an example of having three user 
threads (two PThreads and one thread of the main program) in an OpenMP program. 
The two three threads execute in parallel after the two PThreads are created. 
Each thread calls a function that will enter into
OpenMP threading parallelism. So they all become OpenMP initial threads. How the 
user threads (PThreads in this example) interact with the OpenMP threading mechanisms
in the runtime is up to the implementation. They may share the same OpenMP runtime
instance or each has its own OpenMP runtime instance. 

In the example from Figure~\ref{fig:pthread-omp}, 
one can view this in a two-level threading parallelism: the top level user thread 
parallelism and the bottom level OpenMP threading parallelism.

\begin{figure}[t]
\centering
  \fbox{
 % \lstset{basicstyle=\ttfamily\scriptsize,language=c}
  \lstset{basicstyle=\ttfamily\scriptsize,language=c,numbers=left, %,frame=single,
  deletekeywords={int,if,else,while},
  morekeywords={pragma,omp,target,device,map,
  tofrom,to,from,alloc,parallel,shared,reduction,data,collapse,
  private,dist_iteration,match_range,halo,exchange},
  numbersep=12pt,numberstyle=\color{red}}
  \lstinputlisting{pthread-omp.c}

}
\caption{Three user threads (two PThreads and one main thread) with OpenMP}
  \label{fig:pthread-omp}
\end{figure}

\subsection{Impacts and Discussions}
}

