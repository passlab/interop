Some previous studies have proposed solutions to solve the interoperability and composibility of parallel programming models and libraries. 
For example, Callisto~\cite{Callisto:Harris:2014:CCP:2592798.2592807} and
Lithe~\cite{Lithe:Pan:2009:LEE:1855591.1855602} 
address the interoperability challenge 
through the design of a low-level software layer for common 
resource management underneath multiple parallel runtime systems, such as OpenMP and TBB. % of programming models. 
%They however do not address the algorithm conflicts of different runtime systems. 
The MPC (Multi-Processor Computing) framework~\cite{perache2008mpc} is a unified parallel runtime designed for clusters of large NUMA nodes. 
Through process virtualization and thread-based MPI implementation, MPC enables efficient mixing of MPI, OpenMP, and Pthreads. 
In order to compose multiple simultaneously executing parallel applications, Hugo et al.~\cite{hugo2014composing} extends the starPU runtime system to allow confined execution environments (called scheduling contexts) which can be used to partition computing resources. 
A hypervisor is used to automatically expand or shrink contexts based on runtime resource utilization feedback. 

The MPI endpoints~\cite{Dinan:mpiendpoint_eurompi13}
proposal to the MPI standard relaxes the one-to-one relationship between processes and ranks.
It allows registering a thread in an MPI
process as a MPI communicator rank that is able to independently paraticipate
in message passing operations. There are also efforts of integrating MPI calls as
tasks in a intra-node workstealing runtime~\cite{hcmpi:ipdps13}.

For enable interoperability among distributed HPC programming models, Epperly et al.~\cite{epperly2011composite} proposed a mixed-language environment supporting arbitrary combination of software written in PGAS languages (Co-Array Fortran, UPC, and Titanium) and HPCS languages (Chapel, X10, and Fortress). 
They designed the Scientfic Interface Definition Laguage (SIDL) and Babel Intermediate Object Representation (IOR) as a language-independent object-oriented programming model and type system
to allow software components to share complicated data structures across various languages. 
