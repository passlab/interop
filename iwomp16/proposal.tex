
%\subsubsection{Posibble Solutions and Proposals}
The current support in OpenMP provides limited constrol on oversubscriptions, but are sufficient 
for lots of (if not most of) scenarios if the implemention is available. In the following of this
report, we propose solutions that will provide more
control for oversubscription.
\subsubsection{Change wait policy dynamically to address passive oversubscription}
The idea is to provide a setter and getter for the wait-policy-var
ICV. Compilers from IBM, Cray and Oracle have provide this feature~\cite{ibmwait,craywait,oraclewait}.
There are also different variants of this features depending how much details users can configure
the wait policy.
\paragraph{1: {\sf omp\_set\_wait\_policy(ACTIVE$\vert$PASSIVE)} setter} for the wait-policy-var
ICV with ACTIVE or PASSIVE. This will allow programmer to explicitly change the policy at various 
points during a program's execution. An efficient implementation may use atomic write to the 
global ICV and all threads will react accordingly at some later point of the exectution after the 
ICV is set. So the effects may be delayed.

\paragraph{2: Finer-grained control with new environment variables and setter routine}



There are still some challenges in terms of OpenMP interoperability. 
OpenMP threads that are created by the parallel construct cannot interact with external systems. 
In other words, we are trying to enable the interoperability through flexible communication between OpenMP threads and user threads. 
However, the main goal of this work is to achieve a high level of resource utilization. So, it would be better if OpenMP threads can interact and communicate with user threads. To achieve this goal, we implement four new functions as follows:
\begin{enumerate}
	\item int omp{\_}set{\_}wait{\_}policy(ACTIVE \textbar PASSIVE): 
	set the waiting thread behavior. The function returns the current wait{\_}policy, which could be different from intention of the call depending on the decision made by the runtime. If the value is PASSIVE, waiting threads should not consume CPU power while waiting; while the value is ACTIVE specifies that they should.
	\item int omp{\_}thread{\_}create( ): 
	to give the user the ability to create an OpenMP thread without using \#pragma omp parallel directive, and lets it be a user thread similar to pthread.
	\item int ompe{\_}quiesce( ): 
	to shutdown or unload the OpenMP runtime library.
\end{enumerate}
