OpenMP has become a successful programming model
for developing multi-threaded applications.
However, there are still some challenges in terms of OpenMP's interoperability
with itself and other parallel programming languages and libraries.
In this paper, we explore typical use cases that expose OpenMP's interoperability challenges and
report our proposed solutions for addressing the resource oversubscription issue as the efforts
by the OpenMP Interoperability language subcommittee. 
%which address one of more aspects of OpenMP's interoperability problem.
The solutions include OpenMP runtime routines for
changing at the runtime the waiting policies, which include ACTIVE(SPIN\_BUSY or SPIN\_PAUSE), 
PASSIVE(SPIN\_YIELD or SLEEP), 
of idling threads for improved resource management, and 
routines for supporting contributing OpenMP threads to other thread libraries or tasks. 
Our initial implementations are being done by extending two OpenMP runtime libraries, 
Intel OpenMP (iOMP) and GNU OpenMP (gOMP).
The evaluation results demonstrate the effectiveness of the proposed approach to address the resource 
oversubscription challenge and detailed analysis provide heuristics for selecting an optimal wait policy according
to the oversubscription ratios. 


%This effort is part of OpenMP Interoperabity language s
%Initial results are given for one of the proposed interoperability functionalities 
%, including OpenMP's wait policy,
%a new runtime feature 
%for quiescing the OpenMP thread pool to address oversubscription issue. 
%and an OpenMP-based abstraction for general-purpose OS threads.
