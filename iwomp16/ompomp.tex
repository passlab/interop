Often a parallel machine is shared by multiple users at the same time.
It is very likely that multiple OpenMP programs coexist within the same computation node. 
The OpenMP programs can be compiled by a same OpenMP compiler or different compilers. 
%Currently, an OpenMP implementation, including its runtime, often assumes that it fully occupies the entire computation node,
%without considering the possibility of existence of other running OpenMP programs and their supportive OpenMP runtime instances. 
The coexistence of multiple OpenMP programs raises the following questions:
\begin{itemize}
\item How many OpenMP programs are running on a given node currently?
%\item Which OpenMP runtime system is being used by a running OpenMP program?
\item What resources are used by each of them?
\item How to ensure optimal resource utilization and avoid contentions?
%\item OpenMP environment variables have global impact on all runtime instances. Is this the desired behavior we want?
\end{itemize}

