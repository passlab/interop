Often a parallel machine is shared by users running multiple programs at the same time.
It is very likly that multiple OpenMP programs coexist within the same computation node. 
The OpenMP programs can be compiled by a same OpenMP compiler or different compilers. 
Current, an OpenMP implementation, including its runtime, assumes that it fully occupies the entire computation node,
without considering the possibility of existence of other running OpenMP programs and their supportive OpenMP runtime instances. 
The coexistence of multiple OpenMP runtime instances raises the following questions:
\begin{itemize}
\item How many OpenMP programs are running currently?
\item Which OpenMP runtime system is being used by a running OpenMP program?
\item What resources are used by each of them?
\item How do the concurrently executing OpenMP programs interact with each others to ensure optimal resource utilization?
\item OpenMP environment variables have global impact on all runtime instances. Is this the desired behavior we want?
\end{itemize}

