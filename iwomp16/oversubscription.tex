Oversubscription happens when resources are claimed and held than what is needed.
A program may request more OpenMP threads than the total amount of hardware
threads available when entering a parallel region, which causes excessive competition 
among OpenMP threads for hardware threads and increases runtime overhead. 
When program execution enters into sequential stage after exiting a parallel region, 
those native threads that support the OpenMP threads in the parallel region may still 
alive in the background consuming CPU cycles but not doing actual work for users. This 
will make those hardware threads unavailable to others. 
Oversubscription impact the performance of an applications and the system, 
but should not introduce correctness issue to a program. 

%A typical OpenMP runtime creates a pool of native threads who will execute OpenMP
% parallel regions and/or tasks. 

The two scenarios we mentioned above are the two kinds of oversubscriptions we should try to avoid:
{\bf 1) Active oversubscription}: Claiming or requesting more threads than 
what are available by the system.
{\bf 2) Passive oversubscription}: Thread resources are not released 
after parallel execution. Holding hardware threads after parallel execution may not 
always hurt the performance overall, e.g. it will improve the startup performance of the 
upcoming parallel region. In this category, we are concerning those situations that 
actually impact the performance negatively.

