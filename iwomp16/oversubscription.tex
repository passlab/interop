Oversubscription happens when resources are claimed and held than what are needed.
A program may request more OpenMP threads than the total amount of hardware
threads available when entering a {\sf parallel} region, which causes excessive competition 
among OpenMP threads for hardware cores and increases runtime overhead. 
When program execution enters into sequential stage after exiting a parallel region, 
those native threads that support the OpenMP threads in the parallel region may still 
alive in the background consuming CPU cycles. This 
will make those hardware cores unavailable to others. 
%Oversubscription impact the performance of an applications and the system, 
%but should not introduce correctness issue to a program. 

%A typical OpenMP runtime creates a pool of native threads who will execute OpenMP
% parallel regions and/or tasks. 

The two scenarios we mentioned above are the two kinds of oversubscription we should try to avoid:
{\bf 1) Active oversubscription}: Claiming or requesting more threads than 
what are available by the system.
{\bf 2) Passive oversubscription}: Thread resources are not released 
after parallel execution. It is important to note that holding hardware threads after parallel execution 
may not always hurt the performance overall, e.g. it may improve the start-up performance of the 
upcoming {\sf parallel} region. 
%In this category, we are concerning those situations that actually impact the performance negatively.

