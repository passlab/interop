Oversubscription happens when resources are claimed and held than what is needed.
A program may request more OpenMP threads than the total amount of hardware
threads available when entering a parallel region, which causes excessive competition 
among OpenMP threads for hardware threads and increases runtime overhead. 
When program execution enters into sequential stage after exiting a parallel region, 
those native threads that support the OpenMP threads in the parallel region may still 
alive in the background consuming CPU cycles but not doing actual work for users. This 
will make those hardware threads unavailable to others. 
Oversubscription impact the performance of an applications and the system, 
but should not introduce correctness issue to a program. 

%A typical OpenMP runtime creates a pool of native threads who will execute OpenMP
% parallel regions and/or tasks. 

The two scenarios we mentioned above are the two kinds of oversubscriptions we should try to avoid\footnote{Acknowledgements go to to Jeff Hammond for categorizing the twos and for introducing the two terms}:  
{\bf 1) Active oversubscription}: Claiming or requesting more threads than 
what are available by the system.
{\bf 2) Passive oversubscription}: Thread resources are not released 
after parallel execution. Holding hardware threads after parallel execution may not 
always hurt the performance overall, e.g. it will improve the startup performance of the 
upcoming parallel region. In this category, we are concerning those situations that 
actually impact the performance negatively.

\subsubsection{Current OpenMP}
OpenMP currently (4.0) provides limited support for users to give hints to runtime for
better managing OpenMP threads and native threads, which can be used to help reducing 
the impact of oversubscriptions.
\paragraph{OMP\_DYNAMIC} % environment variable}
This also includes dyn-var ICV, omp\_set\_dynamic and omp\_get\_dynamic runtime routine. OMP\_DYNAMIC could be either
{\bf true} or {\bf false}. When setting the dyn-var ICV to be {\bf true}, user will expect
OpenMP implementation to adjust the number of threads to use for executing parallel
regions in order to optimize the use of system resources.

\paragraph{OMP\_WAIT\_POLICY} %environment variable}
This also include wait-policy-var ICV, but no getter and setter routine. 
OMP\_WAIT\_POLICY could be set as either ACTIVE or PASSIVE. 
The ACTIVE value specifies that waiting threads should mostly be active, consuming processor
 cycles, while waiting. An OpenMP implementation may, for example, make waiting threads spin.
 The PASSIVE value specifies that waiting threads should mostly be passive, not consuming
 processor cycles, while waiting. For example, an OpenMP implementation may make waiting
 threads yield the processor to other threads or go to sleep.
 The details of the ACTIVE and PASSIVE behaviors are implementation defined.
 
 \paragraph{OMP\_THREAD\_LIMIT}
 This also include thread-limit-var ICV and omp\_get\_thread\_limit getter runtime routine.
 The environment variable sets the maximum number of OpenMP threads to use in a contention group by setting the thread-limit-var ICV.
The behavior of the program is implementation defined if the requested value of OMP\_THREAD\_LIMIT is greater than the number of threads an implementation can support. 

OMP\_DYNAMIC and OMP\_THREAD\_LIMIT are approaches to 
addressing active oversubscriptions, and OMP\_WAIT\_POLICY could be used to address 
passive oversubscriptions. 
Since there are no setters for ICVs for OMP\_WAIT\_POLICY 
and OMP\_THREAD\_LIMIT variable in the current standard, 
dynamically changing waiting policy and the maximum number of 
threads at runtime is not available.

%\subsubsection{Posibble Solutions and Proposals}
The current support in OpenMP provides limited constrol on oversubscriptions, but are sufficient 
for lots of (if not most of) scenarios if the implemention is available. In the following of this
report, we propose solutions that will provide more
control for oversubscription.
\subsubsection{Change wait policy dynamically to address passive oversubscription}
The idea is to provide a setter and getter for the wait-policy-var
ICV. Compilers from IBM, Cray and Oracle have provide this feature~\cite{ibmwait,craywait,oraclewait}.
There are also different variants of this features depending how much details users can configure
the wait policy.
\paragraph{1: {\sf omp\_set\_wait\_policy(ACTIVE$\vert$PASSIVE)} setter} for the wait-policy-var
ICV with ACTIVE or PASSIVE. This will allow programmer to explicitly change the policy at various 
points during a program's execution. An efficient implementation may use atomic write to the 
global ICV and all threads will react accordingly at some later point of the exectution after the 
ICV is set. So the effects may be delayed.

\paragraph{2: Finer-grained control with new environment variables and setter routine}


