\documentclass{llncs}
%
\usepackage{graphicx}
\usepackage{placeins}
%\usepackage{hyperref}

\usepackage{comment}
\usepackage{needspace}
\usepackage{url}
\usepackage{xcolor}
\usepackage{color}
\usepackage[ruled]{algorithm2e}

% correct bad hyphenation here
\usepackage{makeidx}  % allows for indexgeneration
\usepackage{fancyvrb}
\usepackage{amsmath, amssymb}
\usepackage{epsfig}
\usepackage{setspace}\usepackage{multirow}
\usepackage{subfigure}

\usepackage{verbatim}
\usepackage{listings}
\usepackage{wrapfig}
%\usepackage{hyperref}
%\usepackage{xltxtra}
\newcommand{\REM}[1]{}
\usepackage{color}
\newcommand {\TODO}[1]{\textcolor{red}{TODO #1}}

\newcommand{\othertm}{\textsuperscript{$\star$}}
\newcommand{\regtm}{\textsuperscript{\textregistered{}}}
\newcommand{\tm}{{\scriptsize\texttrademark{}}}

%\newcommand{\descheader}[1]{{\needspace{3\baselineskip}\vspace{1em}\noindent \fbox{#1}}}


%
%\setlength{\textfloatsep}{0.4cm} 
\begin{document}
%
\mainmatter              % start of the contributions
%
%\title{A Proposal from OpenMP to Address Interoperability Challenges for Parallel Programming} %
\title{A Proposal to OpenMP for Addressing the Resource Oversubscription Challenge}
%Interoperability Challenges of Parallel Programming} %
%\title{A Portable Abstraction for Task Parallelism and Data Movement in Hierarchical Multiprocessors} %
%\titlerunning{Hamiltonian Mechanics}  % abbreviated title (for running head)
%                                     also used for the TOC unless
%                                     \toctitle is used
%
\author{
	Yonghong Yan\inst{1}$^{,}$\inst{5} \and
	Jeff R. Hammond\inst{2}$^{,}$\inst{5} \and
%	Ali Alqazzaz\inst{1} \and
	Chunhua Liao\inst{3} \and
	Alexandre Eichenberger\inst{4}$^{,}$\inst{5} 
}

\institute{
    Department of Computer Science and Engineering, Oakland University, 
    \email{yan@oakland.edu}
    \and
    Parallel Computing Lab, Intel Corp.,
    \email{jeff\_hammond@acm.org}
    \and
    Center for Applied Scientific Computing, Lawrence Livermore National Laboratory,
    \email{liao6@llnl.gov}
    \and
    Thomas J. Watson Research Center, IBM, 
    \email{alexe@us.ibm.com}
    \and
    OpenMP Interoperability Language Subcommittee
}
\maketitle              % typeset the title of the contribution

\begin{abstract}
OpenMP has become a successful programming model
for developing multithreaded applications.
However, there are still some challenges in terms of OpenMP's interoperability
with itself and other parallel programming models.
In this paper, we explore typical use cases to expose OpenMP's interoperability challenges and
propose a number of solutions.
%which address one of more aspects of OpenMP's interoperability problem.
The solutions we consider include OpenMP's wait policy,
a new runtime feature for quiescing the OpenMP thread pool,
and an OpenMP-based abstraction for general-purpose OS threads.

\end{abstract}
%

%\keywords
%Java, GPU, High Performance 
\section{Introduction}
\label{sec:intro}
Large-scale parallel applications are typically developed using multiple parallel programming models in 
a hybrid fashion, e.g. MPI+OpenMP, and using one or mulitple prebuilt scientific and/or platform-specific libraries such as Intel Math Kernel Library (MKL)~\cite{wang2014intel}.
Each of these programming models and libraries often has its own runtime library to handle scheduling of work units and management of computational
and data movement tasks. 
There have been challenging issues for using these models in one application, including 
compatibility issues for compiling and linking, oversubscription of resources at runtime, and the naming conflicts that 
programmers have to create workaround wrappers to deal with.

This paper proposes solutions to the interoperability and composability challenges faced by the OpenMP programming interface, includling those
between multiple OpenMP implementations and/or multiple OpenMP runtime instances of the same implementation, OpenMP 
with native threads (Pthreads and Windows Native threads), OpenMP with other threading languages and libraries such 
as C++11, TBB and Cilkplus, and OpenMP with inter-node programming models such as MPI and PGAS. We 
think the similar challenges exist in other threading based libraries and language implementations, and believe
the solutions we provided  will work for them too.  

Interoperability and composability are closely related, while the interoperability sounds to improve the interactions between multiple models
while composability is meant to improve the modular use of OpenMP with itself and other models. One is from the aspect of system while 
the other is more concerned with software engineering. Both should be considered when developing solutions. 

For parallel programming languages and libraries, most implementations rely on system native threading (Pthreads or Windows Native threads) 
mechanisms to acquires system resources. Each implementation of the same or different programming models has their own mechanism for scheduling
user-level tasks and operations, which is the core part of a runtime system. 
The interoperability challenges are then concerned with how much we want two or more 
runtime instances (for the same or different high-level programming interfaces) to interact with other other for computational resource sharing
and data movement. Thus solutions to these challenges are more in the scope of runtime implementation, than in the level of programming
interfaces and compiler transformations. 
\TODO{I guess there might be language features enabling interoperability. Also are runtime library interfaces part of programming interfaces??}

\REM{
\subsection{What is an OpenMP?}
OpenMP is an implementation model to support the implementation of parallel algorithms. It is primarily designed for shared memory multiprocessors. The goal of OpenMP is to provide a standard and portable API for writing shared memory parallel programs~\cite{dagum1998openmp}. 

OpenMP takes a directive-based approach for supporting parallelism. It consists of a set of directives that may be embedded within a program written in a base language such as Fortran, C, or C++. There are two compelling benefits of a directive-based approach that led to this choice: The first is that this approach allows the same code base to be used for development on both single-processor and multiprocessor platforms; on the former, the directives are simply treated as comments and ignored by the language translator, leading to correct serial execution. The second related benefit is that it allows an incremental approach to parallelism—starting from a sequential program, the programmer can embellish the same existing program with directives that express parallel execution. These directives may be offered within any base language (within the C/C++ languages, directives are referred to as “pragmas”). In addition to directives, OpenMP also includes a small set of runtime library routines and environment variables. These are typically used to examine and modify the execution parameters. The language extensions in OpenMP fall into one of three categories: control structures for expressing parallelism, data environment constructs for communicating between threads, and synchronization constructs for coordinating the execution of multiple threads~\cite{chandra2001parallel}.
\subsection{How does OpenMP work?}
OpenMP uses a fork/join execution model. OpenMP provides two kinds of constructs for controlling parallelism. First, it provides a directive to create multiple threads of execution that execute concurrently with each other. The only instance of this is the parallel directive. Second, OpenMP provides constructs to divide work among an existing set of parallel threads. An instance of this is the do directive. 

An OpenMP program always begins with a single thread of control that has associated with it an execution context or data environment. This initial thread of control is referred to as the master thread. When the master thread encounters a parallel construct, new threads of execution are created along with an execution context for each thread. Each thread has its own stack within its execution context. The execution context for a thread is the data address space containing all the variables specified in the program. Multiple OpenMP threads communicate with each other through ordinary reads and writes to shared variables.
\subsection{OpenMP Runtime Library}
The OpenMP API runtime library routines are external procedures. The return values of these routines are of default kind, unless otherwise specified. Runtime library provides interface to the compiler. The runtime interface is based on the idea that the compiler ``outlines`` code that is to run in parallel into separate functions that can then be invoked in multiple threads. OpenMP provides several runtime library routines to assist you in managing your program in parallel mode. Many of these runtime library routines have corresponding environment variables that can be set as defaults. The runtime library routines enable you to dynamically change these factors to assist in controlling your program. In all cases, a call to a runtime library routine overrides any corresponding environment variable. 

Generally, we can analyze the architecture into two perspectives: the parallelization regions and the data.
\begin{enumerate}
	\item Region perspective.
	We use “parallel” to automatically create multi-threads. And each thread will be executed without order. However, we can use ordered clause to guarantee the code be executed in sequence. There are different types of parallel regions:
	\begin{itemize}
		\item Section means the task is assigned to each thread.
		\item Single means the task is assigned to a random thread.
		\item Master means the task is executed in the master thread.
	\end{itemize}
	For the default parallel regions, we can use “schedule” to design a way to assign tasks to different threads. Generally, we can implement this assignment in three ways:
	\begin{itemize}
		\item Static: equally assign them to n threads.
		\item Dynamic: assign them to the idle thread only.
		\item Guided: implement the dynamic assignment reductively.
	\end{itemize}
	\item Data perspective.
	We have two kinds of variables. Variables that are defined before parallel region are shared among every thread, while those defined in parallel regions can be only accessed by certain threads. We use “threadprivate” to change those shared variables into a private one for each thread. This is done by generating a new private variable for every single thread. For those shared variables, we must pay attention to the data race problem, which defined as two different memory operations are trying to use a same variable, and different execution order may lead to different results. To solve this problem, we can use “critical” or “atomic” directive to guarantee that the data can be only accessed by one thread at a time. We can also set “barriers” to make sure all threads have been executed before starting any new threads. Sometimes the update of certain variables are stored only in registers, we can use “flush” to directly write the data back to memory to make sure that other threads will use the data that already been updated.
\end{enumerate}
}


\section{Use Cases and Challenges of Interoperability}
\label{sec:challenges}
%\TODO{Introduction mentions interoperability between two OpenMP runtimes. But this section does not mention this. That is part of use case #2}
In this section, we describe three use cases of interoperating OpenMP with other parallel system
and discuss the interoperability challenge of OpenMP.
%We then discuss the interoperability challenge, the limitations of the current standard to support 
%interoperability, and the motivation of this work. 
%as part of OpenMP Interoperability Subcommittee. 
%CPU oversubscription and affinity confliction are the two major performance challenges we identified when multiple parallel libraries do not interoperate well with each other. 
\subsection{Three Use Cases} % of OpenMP Interoperability}

In cases where OpenMP is coexisting with at least one other threading model,
we can identify at least three classes of scenarios in which to consider
interoperability: Phased, Concurrent and Nested.
These are illustrated in Figure~\ref{fig:interop-motif}.
Threaded applications that call threaded libraries from a sequential region
are a good example of the Phased motif.
The thread(s) inside of e.g. an MPI library used by an OpenMP application
matches with the Concurrent motif.
Nested can be either OpenMP threads calling application or library
functions that use another threading model, or the other way around.
Of course, in all cases, the coexistence may not be as regular as
the picture describes, but these simple cases are sufficient to reveal
the challenges of interoperating multiple threading models.
\begin{figure}[htb]
\centering
\includegraphics[width=0.5\textwidth]{images/interop-motifs}
\caption{Pictorial description of Phased, Concurrent and Nested
motifs where two threading models must interoperate.
\label{fig:interop-motif}
}
\end{figure}


\subsubsection{Interacting With User threads}
\textbf{Definition}
A user thread is a thread that is not created by OpenMP implementation. A user thread 
could become an OpenMP initial thread. 

The most common example of user threads are POSIX Threads, 
usually referred to as Pthreads with implementation available on most Unix-like 
POSIX-compliant systems. There are also implementations of other thread libraries, 
for example, Windows Native threads and language based threading support such as 
Java threads or others (e.g. qthreads). 

Figure~\ref{fig:pthread-omp} shows an example of having three user 
threads (two PThreads and one thread of the main program) in an OpenMP program. 
The two three threads execute in parallel after the two PThreads are created. 
Each thread calls a function that will enter into
OpenMP threading parallelism. So they all become OpenMP initial threads. How the 
user threads (PThreads in this example) interact with the OpenMP threading mechanisms
in the runtime is up to the implementation. They may share the same OpenMP runtime
instance or each has its own OpenMP runtime instance. 

\begin{figure}[t]
\centering
  \fbox{
 % \lstset{basicstyle=\ttfamily\scriptsize,language=c}
  \lstset{basicstyle=\ttfamily\scriptsize,language=c,numbers=left, %,frame=single,
  deletekeywords={int,if,else,while},
  morekeywords={pragma,omp,target,device,map,
  tofrom,to,from,alloc,parallel,shared,reduction,data,collapse,
  private,dist_iteration,match_range,halo,exchange},
  numbersep=12pt,numberstyle=\color{red}}
  \lstinputlisting{pthread-omp.c}

}
\caption{Three user threads (two Pthreads and one main thread) with OpenMP}
  \label{fig:pthread-omp}
\end{figure}

\subsection{Impacts and Discussions}
A user thread in a program adds additional level(s) in the overall ``threading''
hierarchy of a program. Those additional levels could be on top of OpenMP threading 
mechanism when a user thread becomes an OpenMP initial thread that creates OpenMP thread
parallelism, or beneath the OpenMP threading mechanism when an OpenMP thread spawns 
a user thread, or the the mix of both. In the example from Figure~\ref{fig:pthread-omp}, 
one can view this in a two-level threading parallelism: the top level user thread 
parallelism and the bottom level OpenMP threading parallelism.

These additional levels of threading increase the complexity of a program, both for 
users in the aspect of reasoning the parallel and synchronization behavior of a program, 
and also for the implementation and runtime system in terms of resource management and 
interactions. Adding to the complexity is the facts that a user thread may be created 
through a call to a library function whose paralelism (OpenMP) behavior is not known to 
the callee. Typical issues for example: 
\begin{itemize}
\item Does each user thread use the same OpenMP runtime libraries or not? 
	If not using the same library, how to handle symbol name 
	conflicts of two more different OpenMP runtime libraries. 
\item For user threads that use the same OpenMP runtime library, does the user threads each create its own runtime instance or they share one?
\item For user threads each of which has its own runtime instance (from the same or 
	different runtime library), how to coordinate the resource management among those
	runtime instances to address such issues as oversubscriptions and the affinity
	between user threads?
\end{itemize}

It is important to note that approaches to address those issues are very implementation
dependent, requiring protocol and agreement in the runtime behavior and/or interfaces 
of different OpenMP implementations. It may not be realistic to solve some of the issue
from the language standard, and should be left to users to deal with them. In this
aspect, we still hope this report could provide userful information and practices 
for users. 



%\subsubsection{Interacting with Other OpenMP Programs}
%Often a parallel machine is shared by users running multiple programs at the same time.
It is very likly that multiple OpenMP programs coexist within the same computation node. 
The OpenMP programs can be compiled by a same OpenMP compiler or different compilers. 
Current, an OpenMP implementation, including its runtime, assumes that it fully occupies the entire computation node,
without considering the possibility of existence of other running OpenMP programs and their supportive OpenMP runtime instances. 
The coexistence of multiple OpenMP runtime instances raises the following questions:
\begin{itemize}
\item How many OpenMP programs are running currently?
\item Which OpenMP runtime system is being used by a running OpenMP program?
\item What resources are used by each of them?
\item How do the concurrently executing OpenMP programs interact with each others to ensure optimal resource utilization?
\item OpenMP environment variables have global impact on all runtime instances. Is this the desired behavior we want?
\end{itemize}



\subsubsection{Interacting with Parallel Libraries and Language Concurrency Features}
It is now very common that an application uses multiple parallel libraries at the same time, which could be developed 
using OpenMP, TBB, Cilkplus, C++11, and other parallel libraries. 
%In Figure~\ref{fig:cholesky}, 

For example, a Cholesky Factorization~\cite{intertwine} may use OpenMP 
tasking and BLAS operations provided by Intel MKL library,
%~\footnote{The program was provided by Xavier Teruel from BSC demonstrating the interoperability problem for the INTERWinE project.}, 
which is a parallel math library. The runtime will then need to coordinate the two
parallel runtimes if the twos do not integrate, e.g. each has its own instance during the program execution. Another use case is the existence of multiple OpenMP runtime instances from either same or different libraries. These runtime instances could be created by user threads that launch into
OpenMP operations. 
%\begin{figure}[h!]
%  \centering
%      \includegraphics[width=0.75\textwidth]{images/cholesky}
%      \caption{The Mixed Use of OpenMP Tasking and Intel MKL Library for Cholesky Factorization~\cite{intertwine}}
% \label{fig:cholesky}
%\end{figure}



\subsubsection{Interoperability with Inter-node Model, e.g. MPI}
Hybrid parallel programming in the form of internode+intranode, e.g. MPI+X model are widely used  
for HPC, which is the typical example for the Concurrent class. 
This hybrid approach reflects the two-level hierarchy of hardware parallelism 
in current HPC systems, in which network connects many highly parallel nodes.
Interoperability between inter- and intra-node APIs such as MPI+OpenMP 
has long been a productivity and composability goal within
the HPC community. We however still have not agreed on a 
standard solution from either of the two communities. 




%In this
%aspect, we still hope this paper could provide userful information and practices 
%for users. 
%\subsubsection{OpenMP with native threads}
%\subsubsection{OpenMP with TBB/MKL}
%\subsubsection{OpenMP with parallel scientific library, such as MKL}
%\subsubsection{Linking libraries and objects built with different OpenMP compilers}
%\subsubsection{OpenMP with inter-node model, e.g. MPI}


\subsection{Issues with No or Poor Interoperability}
In the use cases for intra-node interoperability, 
the coexistence of multiple user threads and other runtime instances
adds additional level(s) in the overall ``threading''
hierarchy of a program. % Those additional levels could be on top of OpenMP threading mechanism when 
These additional levels of threading increase the complexity of a program for 
users and complicate the reasoning of parallel and synchronization behaviors of parallel tasks. 
More critically, it introduces at least two performance issues
since most OpenMP implementations ignore their existence % user threads or other parallel runtime instance 
in the decision making for runtime scheduling and resource management. 
%when mapping computation and data to CPUs and memory.  

\subsubsection{CPU Oversubscription}
Oversubscription happens when resources are claimed and held than what are needed.
A program may request more OpenMP threads than the total amount of hardware
threads available when entering a {\sf parallel} region, which causes excessive competition 
among OpenMP threads for hardware cores and increases runtime overhead. 
When program execution enters into sequential stage after exiting a {\sf parallel} region, 
those native threads that support the OpenMP threads in the parallel region may still 
alive in the background consuming CPU cycles. This 
will make those hardware cores unavailable to others. 
%Oversubscription impact the performance of an applications and the system, 
%but should not introduce correctness issue to a program. 

%A typical OpenMP runtime creates a pool of native threads who will execute OpenMP
% parallel regions and/or tasks. 

The two scenarios we mentioned above are the two kinds of oversubscription we should try to avoid:
{\bf 1) Active oversubscription}: Claiming or requesting more threads than 
what are available by the system.
{\bf 2) Passive oversubscription}: Thread resources are not released 
after parallel execution. It is important to note that holding hardware threads after parallel execution 
may not always hurt the performance overall, e.g. it may improve the start-up performance of the 
upcoming {\sf parallel} region. 
%In this category, we are concerning those situations that actually impact the performance negatively.



\subsubsection{Conflicting Thread Affinity}
Besides oversubscribing CPU cores by an OpenMP program, 
memory are another kind of resources that should be coordinately allocated and managed among
multiple parallel runtimes. %about how resources are utilized. 
When the OpenMP runtime binds threads data to 
certain memory places (cache or NUMA region) that are already occupied by 
the affinity requests of another runtime, thread affinity conflict happens.  
Such memory overlaps cause excessive cache or memory spills or relocation of data
to farther places, resulting increased memory access latency because of false sharing or 
poor locality of thread and its data. %the places.
The affinity conflicts can happen even the total number of threads requested does not exceed the number of hardware threads available.
%Obviously, conflicting thread affinity may adversely impact the performance of programs involved. 



\subsection{Limitation of Interoperability Support in the Standard}
The current OpenMP standard (4.5) provides limited support for users to give hints to runtime for 
better managing OpenMP threads and native threads, which can be used to help reducing 
the impact of oversubscription.
First, the {\sf OMP\_DYNAMIC} group of constructs, which includes the environment variable, 
the {\sf dyn-var} ICV, {\sf omp\_set\_dynamic} and {\sf omp\_get\_dynamic} runtime routine,
indicates the OpenMP implementation to adjust the number of threads to use for executing parallel
regions in order to optimize the use of system resources. 
This approach address only the active oversubscription issue. 
%. OMP\_DYNAMIC could be either
%{\bf true} or {\bf false}. When setting the dynvar ICV to be {\bf true}, user will 

%\paragraph{OMP\_WAIT\_POLICY} %environment variable}
Second, the {\sf ACTIVE} setting for the {\sf OMP\_WAIT\_POLICY} and its associated {\sf wait-policy-var} ICV, 
dictates the waiting thread in sequential region to be actively waiting, which consume processor cycles.
The {\sf PASSIVE} setting allows those threads to yield CPUs for others to use, which addresses the passive 
oversubscription issue. The standard however only allows one time setting of the variable when the program starts, thus 
preventing the dynamic adjustment of thread waiting behavior during the execution. 

Thirdly, the {\sf OMP\_THREAD\_LIMIT} environment, {\sf thread-limit-var} ICV and the {\sf omp\_get\_thread\_limit} getter give users an option to  
 set the maximum number of OpenMP threads to use in a contention group when the program starts, addressing oversubscription issues in certain
 degree. It however does not provide setter to adjust the upper bound of the threads for an OpenMP program during program execution, limiting
 its actually usage in real applications. 
% by setting the threadlimitvar ICV.
%The behavior of the program is implementation defined if the requested value of OMP\_THREAD\_LIMIT is greater than the number of threads an implementation can support

%\subsubsection{Global Impact of Environment Variables}
The interoperability and composability of OpenMP programs are also limited by the global impact of OpenMP environment variables.
Currently, the OpenMP specification has an implicit assumption that a single OpenMP program is running on a computation node at any given time.
So the provided global environment variables should work properly to set the internal control variables that affect the execution
of the single OpenMP program. 
However, this causes problem when multiple OpenMP programs are running simultaneously on a computation node. 
Environment variables affecting resource allocation may uniformly impact all OpenMP programs, which is often not desired.
%For example, unless runtime library routines are explicitly used to customize each program's choices, \lstinline{OMP_NUM_THREADS}, \lstinline{OMP_PROC_BIND}, \lstinline{OMP_DEFAULT_DEVICE} may 
%make all OpenMP runtime instances use the same number of threads, the same affinity, and the same accelerator device, respectively. 



As the combinations of different parallel programming APIs in one application
at different system levels are becoming more
practical solution than creating a single unified and 
comprehensive model, it becomes urgent for OpenMP to enhance its 
interoperability support.
%and to address the two important performance issues with better solutions 
%than what the current standard offers. The proposal in this paper provides 
%solutions to the CPU oversubscription issue and a comprehensive solution is the ongoing effort 
%that will provide an extensible set of interfaces for addressing other interoperability issues. 



%It is important to note that approaches to address those issues are very implementation
%dependent, requiring protocol and agreement in the runtime behavior and/or interfaces 
%of different OpenMP implementations. It may not be realistic to solve some of the issue
%from the language standard, and should be left to users to deal with them. 

%Thus the current standard provide minimum support for interoperability, and it is becoming urgent
%to provide more support for this challenge since
%in the short to medium term. A




 \REM{
%but no getter and setter routine. 
%OMP\_WAIT\_POLICY could be set as either ACTIVE or PASSIVE. 
%The ACTIVE value specifies that waiting threads should mostly be active, consuming processor
% cycles, while waiting. An OpenMP implementation may, for example, make waiting threads spin.
% The PASSIVE value specifies that waiting threads should mostly be passive, not consuming
% processor cycles, while waiting. For example, an OpenMP implementation may make waiting
% threads yield the processor to other threads or go to sleep.
% The details of the ACTIVE and PASSIVE behaviors are implementation defined.
 
 \paragraph{OMP\_THREAD\_LIMIT}
 This also include thread-limit-var ICV and omp\_get\_thread\_limit getter runtime routine.
 The environment variable sets the maximum number of OpenMP threads to use in a contention group by setting the thread-limit-var ICV.
The behavior of the program is implementation defined if the requested value of OMP\_THREAD\_LIMIT is greater than the number of threads an implementation can support

OMP\_DYNAMIC and OMP\_THREAD\_LIMIT are approaches to
addressing active oversubscription, and OMP\_WAIT\_POLICY could be used to address
passive oversubscription. 
Since there are no setters for ICVs for OMP\_WAIT\_POLICY
and OMP\_THREAD\_LIMIT variable in the current standard, 
dynamically changing waiting policy and the maximum number of
threads at runtime is not available.
	
Adding to the complexity is the fact that a user thread may be created 
through a call to a library function whose parallelism (OpenMP) behavior is not known to 
the callee. Typical interoperability issues include the following examples: 
\begin{itemize}
\item Does each user thread use the same OpenMP runtime library or not? 
	If not using the same library, how to handle symbol name 
	conflicts of two more different OpenMP runtime libraries? 
\item For user threads that use the same OpenMP runtime library, does the user threads each create its own runtime instance or they share one?
\item For user threads each of which has its own runtime instance (from the same or 
	different runtime library), how to coordinate the resource management among those
	runtime instances to address such issues as oversubscription and the affinity
	between user threads?
\end{itemize}
%\TODO{What if a user thread does not use any OpenMP runtime at all? }
}

%\newpage

\section{Interoperability Proposal}
\label{sec:proposal}

%\subsubsection{Posibble Solutions and Proposals}
The current support in OpenMP provides limited constrol on oversubscriptions, but are sufficient 
for lots of (if not most of) scenarios if the implemention is available. In the following of this
report, we propose solutions that will provide more
control for oversubscription.
\subsubsection{Change wait policy dynamically to address passive oversubscription}
The idea is to provide a setter and getter for the wait-policy-var
ICV. Compilers from IBM, Cray and Oracle have provide this feature~\cite{ibmwait,craywait,oraclewait}.
There are also different variants of this features depending how much details users can configure
the wait policy.
\paragraph{1: {\sf omp\_set\_wait\_policy(ACTIVE$\vert$PASSIVE)} setter} for the wait-policy-var
ICV with ACTIVE or PASSIVE. This will allow programmer to explicitly change the policy at various 
points during a program's execution. An efficient implementation may use atomic write to the 
global ICV and all threads will react accordingly at some later point of the exectution after the 
ICV is set. So the effects may be delayed.

\paragraph{2: Finer-grained control with new environment variables and setter routine}



There are still some challenges in terms of OpenMP interoperability. 
OpenMP threads that are created by the parallel construct cannot interact with external systems. 
In other words, we are trying to enable the interoperability through flexible communication between OpenMP threads and user threads. 
However, the main goal of this work is to achieve a high level of resource utilization. So, it would be better if OpenMP threads can interact and communicate with user threads. To achieve this goal, we implement four new functions as follows:
\begin{enumerate}
	\item int omp{\_}set{\_}wait{\_}policy(ACTIVE \textbar PASSIVE): 
	set the waiting thread behavior. The function returns the current wait{\_}policy, which could be different from intention of the call depending on the decision made by the runtime. If the value is PASSIVE, waiting threads should not consume CPU power while waiting; while the value is ACTIVE specifies that they should.
	\item int omp{\_}thread{\_}create( ): 
	to give the user the ability to create an OpenMP thread without using \#pragma omp parallel directive, and lets it be a user thread similar to pthread.
	\item int ompe{\_}quiesce( ): 
	to shutdown or unload the OpenMP runtime library.
\end{enumerate}


\section{Implementation and Evaluation}
\label{sec:implementation}
The implementation is performed in both Intel OpenMP runtime (iOMP) Version 20160322 and GNU Compiler Runtime (GOMP) version 6.1.0. 
we highlight the implementation and evaluation of some of the functions in Level 2 and 3: 
{\sf omp\_set\_wait\_policy}, {\sf omp\_quiesce} and {\sf omp\_thread\_create/exit/join}. 

For iOMP, the implementation of {\sf omp\_set\_wait\_policy} leverages the available 
implementation of suspending and resuming a thread used for the barrier implementation. Setting the 
wait policy of a thread to the {\sf SLEEP} policy is implemented by setting thread-specific variables. For
supporting changing policies to either {\sf SPIN} or {\sf YIELD}, the implementation needs to include codes for 
waking up those threads who are sleeping by using the provided thread suspension and resumption mechanisms.  
The thread suspension and resumption functionalities are implemented using pthread condition variable and mutex and 
we validate the effectiveness of our implementation by checking the kernel state of the program state when being 
executed.

The thread behaves in the waiting state according to the policies specified by call to the {\sf omp\_set\_wait\_policy} or the default value from the wait-policy-var ICV. Below is the algorithm for the waiting loop.
\begin{algorithm}[ht]
	\small
    \caption{The waiting loop of threads that are not computing}

 \While{fork\_barrier\_not\_released}{
   thread\_state = omp\_thread\_state\_SPIN\;
   \If{unfinished tasks} {
	   execute\_task ( ... )\;
   }

   \If{program finished} {
	   break\;
   }

   \If{wait\_policy ==  omp\_thread\_state\_SPIN} {
	   continue\;
   }

   \If{wait\_policy ==  omp\_thread\_state\_YIELD} {
   	   thread\_state = omp\_thread\_state\_YIELD\;
	   spins = thread\_yield\_spin\_time\;
	   \While{program not finished and spins is greater than 0} {
		sched\_yield()\;
		spins -= 2\;
	   }
	   continue\;
   }
   thread\_state = omp\_thread\_state\_SLEEP\;
   thread\_suspend()\;

   
   \If{program finished} {
	   break\;
   }
   }
 \label{algo:worker_sched}
\end{algorithm} 


The {\sf omp\_quiesce} implementation leverages the support for {\sf omp\_set\_wait\_policy} except that it 
should change the behavior of all the threads when it is called with SLEEP argument. 
%{\sf ACTIVE} and {\sf PASSIVE}
%environment setting for {\sf OMP\_WAIT\_POLICY} and the corresponding ICV, the use of turnaround and throughput mode for OpenMP execution, the 
%use of {\sf KMP\_BLOCKTIME} and {\sf kmp\_set\_blocktime} for setting the waiting time before putting a thread to sleep~\cite{iccmanual}, 
The implementation of {\sf omp\_quiesce(omp\_thread\_state\_KILL)} is a wrapper of 
the {\sf \_\_kmp\_internal\_end\_fini} internal API which shuts down the runtime system~\cite{iccmanual}. 


The implementation of  {\sf omp\_thread\_create/exit/join} needs to work around the need of a team when creating a thread, and the requirement 
of a fork barrier and join barrier when assigning any work to a thread in the runtime. An internal team object is created when creating a thread 
using the {\sf omp\_thread\_create} function. The implementation will first check the thread pool to find a waiting thread. If there is no thread
in the pool, it creates a native thread (PThread in Unix/Linux) launching into the runtime loop. The thread will be released for performing
user specified routine by the fork barrier. But it will skip the join barrier since it does not form a team with any other thread. 
The {\sf omp\_thread\_join} function is realized using a join counter. 

The total amount changes are about 200 lines of code, thus requiring small amount of efforts.  

\REM{
In general, to implement those four functions, we follow the three steps:
\begin{itemize}
	\item Define this function in file “kmp{\_}csupport.c”, 
	write down the implementation.
	\item Declare this function in file “kmp.h”, 
	using “KMP{\_}EXPORT” in front the declaration.
	\item Export this function in file “dllexports”, 
	assign a unique ID for this function.
\end{itemize}

\begin{enumerate}
	\item void omp{\_}quiesce()
	
	The purpose of this function is to shutdown or destroy all OpenMP threads in the thread pool. We have implemented it, as shown in Figure~\ref{omp:quiesce}, by using the Intel internal call to {\_}kmp{\_}internal{\_}end{\_}fini, which unloads the runtime library. Then, we have to register the master thread again so it can generate team of threads later when needed. This can be done by calling the {\_}kmp{\_}get{\_}global{\_}thread{\_}id{\_}reg( ).
	
	\begin{figure}
		\centering
		\includegraphics[width=0.7\textwidth] {images/omp_quisce}
		\caption{omp\_quiesce}
		\label{omp:quiesce}
	\end{figure}
	
	\item void omp{\_}set{\_}wait{\_}policy(PASSIVE \textbar ACTIVE)
	
	The idea of this function is to set the waiting thread behavior. PASSIVE value means that waiting threads should not consume CPU power while waiting. In other words, the OpenMP runtime system will put them into a sleep mode. On the other hand, ACTIVE value means that waiting threads should keep asking the CPU for work to do. The intention of doing this function is to measure the differences in performance between these different modes. 
	The implementation of this function is done by using the internal {\_}kmp{\_}stg{\_}parse{\_}wait{\_}policy as shown in Figure~\ref{omp:set_wait_policy}. The current OpenMP runtime system uses the library{\_}turnaround to indicate the ACTIVE mode and library{\_}throughput to indicate the PASSIVE mode. We pass an integer as its parameter. If it equals to 0, we set the wait policy to be passive, otherwise, active. We found a variable named “{\_}kmp{\_}library” in the environment setting file which has four different status for the waiting policy. So, we change this value accordingly, then we call a function “{\_}kmp{\_}aux{\_}set{\_}library” to set the changed value to the OpenMP environment.
	
	\begin{figure}
		\centering
		\includegraphics[width=0.7\textwidth] {images/omp_set_wait_policy}
		\caption{omp\_set\_wait\_policy}
		\label{omp:set_wait_policy}
	\end{figure}
	
	\item int omp{\_}thread{\_}create()
	
	The purpose of this function is to give the user the ability to create an OpenMP thread without using \#pragma omp parallel directive, and lets it be a user thread similar to pthread. The implementation of this function is shown in Figure~\ref{omp:create_thread}.
	So, we are creating one thread to execute the passed function. If there are enough available threads in the thread pool, we will get one thread from the thread pool and assign the task to it. If no thread is available in the thread pool, we create a new thread to execute this task, and then put the new thread back into the thread pool after completing its job. 
	
	\begin{figure}
		\centering
		\includegraphics[width=0.7\textwidth] {images/omp_create_thread}
		\caption{omp\_create\_thread}
		\label{omp:create_thread}
	\end{figure}
	
\end{enumerate}
}

\subsection{Evaluation}
\label{sec:results}
The evaluation was performed on a machine with 2 Intel(R) Xeon(R) CPU E5-2699 for total 36 cores supporting total 72 threads. We use the latest
LLVM compiler version 3.8.0 for evaluating iOMP implementation. Test programs were designed to evaluate the overhead of these functions. 

\subsubsection{{\sf omp\_quiesce}}
Figure~\ref{omp:quiesce_evaluation} shows the design of the evaluation. The test intended to measure the overhead of starting 
up an OpenMP runtime and shuting down a runtime by using {\sf omp\_quiesce}, as compared to creating a parallel region. 
Figure~\ref{omp:quiesce_results} shows that results. 
%of the running time of all variables (startup{\_}quisece, parallel, and quiesce) increase as we increase the number of threads used. 
The cost of creating a parallel region is very light because of the use of internal hot teams maintained by the runtime. The cost of shutdowning a runtime is about half of the cost of starting a runtime, but about 4 to 10 times of the cost of parallel region creation. 



%because the OpemMP just creates that once. Then, it puts them in a global thread pool to be used next time needed. However, the time cost represented by the quiesce term refers to the time required to shutdown the whole runtime library. In other words, after each parallel region we remove all threads in the global thread pool. Finally, the startup{\_}quiesce term implies the time required to initialize the parallel region and the time taken to shutdown the runtime library. 

%\vspace{-0.6cm}
\begin{figure}[ht]
\subfigure[Test Design]{
		\includegraphics[width=0.6\textwidth] {images/quiesce_evaluation}
		\label{omp:quiesce_evaluation}
}
\subfigure[Evaluation Results]{
		\includegraphics[width=0.6\textwidth] {images/quiesce_results}
		\label{omp:quiesce_results}
}
\caption[]{{\sf omp\_quiesce} Evaluation}
\label{fig:quiesce}
\end{figure}
%\vspace{-0.4cm}


\REM{

\begin{enumerate}
	\item void omp{\_}quiesce()
	\item void omp{\_}set{\_}wait{\_}policy(PASSIVE \textbar ACTIVE)
	
	We need to create two processes since each process will only maintain and share one thread pool. For those two process, each task is execute using 1s, and we need to create enough threads to make full use of the calculation power of one CPU. We tested it in three cases: passive, active, and quiesce/restart the runtime environment. Figure~\ref{omp:wait_policy_evaluation} shows the design of the evaluation.
	
	\begin{figure}
		\centering
		\includegraphics[width=0.7\textwidth] {images/wait_policy_evaluation}
		\caption{waiting policy evaluation}
		\label{omp:wait_policy_evaluation}
	\end{figure}
	
	As Figure~\ref{omp:wait_policy_results} below shows, there is no a big difference between the two behaviors. The reason is that the OpenMP uses only one global thread pool for all OpenMP threads created by multiple pthreads. So, the small difference comes from the time required to awake a sleeping thread. By doing this experiment, we have understand more about the way that OpenMP deals with the thread pool. 
	
	\begin{figure}
		\centering
		\includegraphics[width=0.7\textwidth] {images/wait_policy_results}
		\caption{Waiting Policy Results}
		\label{omp:wait_policy_results}
	\end{figure}
	
	
	\item int omp{\_}thread{\_}create()
	
	We compared this function with creating pthread to execute a list of tasks. So, for this function we have tested it in two different ways. Figure~\ref{omp:create_evaluation} shows the design of the evaluation. For the first way, we put different number of tasks in one parallel region, so that every “omp{\_}thread{\_}create()” or “pthread{\_}create()” function will be run in parallel. On the other hand, we use different iterations to execute the “omp{\_}thread{\_}create()” or “pthread{\_}create()” functions in sequence, and compare the running time. 
	
	\begin{figure}
		\centering
		\includegraphics[width=0.7\textwidth] {images/create_evaluation}
		\caption{creating thread evaluation}
		\label{omp:create_evaluation}
	\end{figure}
	
	Figure~\ref{omp:create_results_parallel2} and Figure~\ref{omp:create_results_parallel} show the result of the first approach (execute in parallel). It clearly shows that there is almost no differences between them. This is might be because that we are doing it inside the parallel region.However, Figure~\ref{omp:create_results_sequence2} and Figure~\ref{omp:create_results_sequence} show the result of the second approach (execute in sequence). They show that omp{\_}thread{\_}create() gives a better performance that pthread{\_}create( ). So, it would be a good feature if the user can do this instead of creating another pthread.

% I do not think we need these.  The performance should be relatively obvious
% and in any case should not matter.  The purpose is to create an OS thread
% that the OpenMP runtime is aware of, and otherwise it should be the thinnest
% possible wrapper about the native OS thread library.
% Because of this, I do not see how omp_thread_create can be faster than
% pthread_create.
%                       - Jeff
\begin{comment}
	\begin{figure}
		\centering
		\includegraphics[width=0.7\textwidth] {images/create_results_parallel2}
		\caption{Results of omp\_thread\_create in parallel}
		\label{omp:create_results_parallel2}
	\end{figure}
	\begin{figure}
		\centering
		\includegraphics[width=0.7\textwidth] {images/create_results_parallel}
		\caption{Results of omp\_thread\_create in parallel}
		\label{omp:create_results_parallel}
	\end{figure}
	\begin{figure}
		\centering
		\includegraphics[width=0.7\textwidth] {images/create_results_sequence2}
		\caption{Results of omp\_thread\_create in sequence}
		\label{omp:create_results_sequence2}
	\end{figure}
	\begin{figure}
		\centering
		\includegraphics[width=0.7\textwidth] {images/create_results_sequence}
		\caption{Results of omp\_thread\_create in sequence}
		\label{omp:create_results_sequence}
	\end{figure}
\end{comment}

\end{enumerate}

}


 
%\section{Hierarchical Place Trees (HPT) Model} 
%\label{sec:model}
%\input{modeling}

%\section{Programming Interface and Implementation}
%\label{sec:interface}
%\input{interface}

%\vspace{-0.8cm}
%\section{Preliminary Experimental Results}
%\label{sec:eva}
%\input{eva}

\section{Related Work}
\label{sec:related}
Previous studies address interoperability and composability of parallel programming models and libraries 
from different aspect and for different programming models. 
Tian et al.~\cite{tian2003compiler} explored interoperability between OpenMP threads and system threads in the Intel OpenMP compiler.
For ease of use for programmers, they decided not to share thread identifiers between system threads and their OpenMP parent
and not to share {\sf threadprivate} variables among system threads.
%Also, a forked system thread calling \lstinline{omp_in_parallel()} would return false, even if it is created by an OpenMP thread. 
Callisto~\cite{Callisto:Harris:2014:CCP:2592798.2592807} and
Lithe~\cite{Lithe:Pan:2009:LEE:1855591.1855602} 
address the interoperability challenge 
through the design of a low-level software layer for common 
resource management underneath multiple parallel runtime systems, such as OpenMP and TBB. % of programming models. 
%They however do not address the algorithm conflicts of different runtime systems. 
In order to compose multiple simultaneously executing parallel applications, Hugo et al.~\cite{hugo2014composing} extends the starPU runtime system to allow confined execution environments (called scheduling contexts) which can be used to partition computing resources. 
A hypervisor is used to automatically expand or shrink contexts based on runtime resource utilization feedback. 

The MPC (Multi-Processor Computing) framework~\cite{perache2008mpc} is a unified parallel runtime designed for clusters of large NUMA nodes. 
Through process virtualization and thread-based MPI implementation, MPC enables efficient mixing of MPI, OpenMP, and PThreads. 
The MPI endpoints~\cite{Dinan:mpiendpoint_eurompi13}
proposal to the MPI standard relaxes the one-to-one relationship between processes and ranks.
It allows registering a thread in an MPI
process as a MPI communicator rank that is able to independently participate
in message passing operations. There are also efforts of integrating MPI calls as
tasks in a intra-node workstealing runtime~\cite{hcmpi:ipdps13}.

To enable interoperability among distributed HPC programming models, Epperly et al.~\cite{epperly2011composite} proposed a mixed-language environment supporting arbitrary combination of software written in PGAS languages (Co-Array Fortran, UPC, and Titanium) and HPCS languages (Chapel, X10, and Fortress). 
They designed the Scientific Interface Definition Language (SIDL) and Babel Intermediate Object Representation (IOR) as a language-independent object-oriented programming model and type system
to allow software components to share complicated data structures across various languages. 

%\FloatBarrier
\section{Conclusions and Future Work}
\label{sec:conclusion}
In conclusion, we have seen that there are many features can be added to the current 
OpenMP Runtime Library in order to improve the OpenMP interoperability. One feature is 
that allowing the user to create a new OpenMP thread and assign a task to it instead 
of creating new user thread. We have implement a function to allow users to get one 
thread from the existing thread pool is any threads are available, and assign one task 
to this thread, this helps to take advantage of the OpenMP thread pool and won’t need 
to create a new thread to work on it, which helps to save the memory usage and speed up the runtime.

We have studied the waiting policy of the OpenMP and how the current OpenMP Runtime System deals with the thread pool. Considering there are two waiting policies, one called throughput (passive), which is designed to make the program aware of its environment (that is, the system load) and to adjust its resource usage to produce efficient execution in a dynamic environment. While the other one called turnaround (active), which is designed to keep active all of the processors involved in the parallel computation in order to minimize the execution time of a single job. We cannot simply say which one is better than the other, it depends one the executing environment. When setting the wait policy to be passive, after a certain period of time has elapsed, the useless thread will stop waiting and sleep. Thus active mode may be better for high-density of OpenMP tasks. While, a passive mode with a small blocktime value may offer better overall performance if your application contains non-OpenMP threaded code that executes between parallel regions. 

In addition, we have implemented a new function to shutdown the whole runtime library when exiting the parallel region. Since all threads are maintained in the same thread pool, quiesce will reap every threads to free the memory, which sometimes help to clear the runtime environment when the task density is lower and we don’t need to wake up most of the thread in the thread pool. However, when entering new parallel regions, we need to make sure that we register the current working thread as our root thread, so that new runtime environment can be built on it. It cost time to restart another parallel region, thus works slower when lots of tasks in the task queue.

As a future work, we should continue adding more functions to the existing runtime system to improve the OpenMP interoperability, such as omp{\_}attach/omp{\_}detach, omp{\_}exit/omp{\_}join. By doing this, we could have a better OpenMP runtime library that optimizes the resources utilization.
%\vspace{-0.1in}
\section*{Acknowledgments}
%The solutions are based on the various discussions from 
We thank members from the OpenMP Interoperability language subcommitt and the language 
commitee in general for providing insightful comments of the design. 
%We particularly appreciate Alexandre Eichenberger from IBM for . 
We are also grateful to Terry Wilmarth and Brian Bliss from Intel for providing information for our 
implementation. 
This material is based upon work supported by the National
Science Foundation under Grant No. SHF-1409946 and SHF-1551182. 
% to stay within page limit, remove for now
%\scriptsize{
%    ~\\
%    \noindent\othertm{}Other names and brands may be claimed as property of others.
%
%    \noindent
%    Intel and Xeon are trademarks of Intel Corporation in the U.S. and/or other countries.
%    Software and workloads used in performance tests may have been optimized
%    for performance only on Intel microprocessors.  Performance tests, such as
%    SYSmark and MobileMark, are measured using specific computer systems,
%    components, software, operations and functions.  Any change to any of those
%    factors may cause the results to vary.  You should consult other information
%    and performance tests to assist you in fully evaluating your contemplated
%    purchases, including the performance of that product when combined with
%    other products.  For more information go to \url{http://www.intel.com/performance}.
%}

\small{
\bibliographystyle{plain}
%IEEEtran}
%\nocite{*}
\bibliography{IEEEabrv,interop}
}
\end{document}
