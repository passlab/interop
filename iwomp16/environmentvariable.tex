The interoperability and composability of OpenMP programs are also limited by the global impact of OpenMP environment variables.
Currently, the OpenMP specification has an implicit assumption that a single OpenMP program is running on a computation node at any given time.
So the provided global environment variables should work properly to set the internal control variables that affect the execution
of the single OpenMP program. 
However, this causes problem when multiple OpenMP programs are running simultaneously on a computation node. 
Environment variables affecting resource allocation may uniformly impact all OpenMP programs, which is often not desired.
%For example, unless runtime library routines are explicitly used to customize each program's choices, \lstinline{OMP_NUM_THREADS}, \lstinline{OMP_PROC_BIND}, \lstinline{OMP_DEFAULT_DEVICE} may 
%make all OpenMP runtime instances use the same number of threads, the same affinity, and the same accelerator device, respectively. 
