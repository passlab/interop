\documentclass{article}
\usepackage{graphicx}
\DeclareGraphicsExtensions{.pdf}

\headheight 0in
\oddsidemargin 0in
\evensidemargin  0in
\topmargin  -.25in
\textwidth 6.5in
\textheight 9in
%\title{OMPinterop: Interoperability and Composability of OpenMP\textsuperscript{\textregistered}}
\title{Interoperability and Composability of OpenMP\textsuperscript{\textregistered}}
\author{Yonghong Yan\thanks{cOMPunity}, 
%John Mellor-Crummey\thanks{Rice University}, 
%Martin Schulz\thanks{Lawrence Livermore National Laboratory},
\\~\\
%Nawal Copty\thanks{Oracle}, 
%Jim Cownie\thanks{Intel},
% John DelSignore\thanks{Rogue Wave}, 
%Robert Dietrich\thanks{TU Dresden, ZIH},
%Xu Liu\hbox to 0in{$^\dagger$\hss},
%Eugene Loh\hbox to 0in{$^\S$\hss}, 
%Daniel Lorenz\thanks{J\"{u}lich Supercomputer Center}, 
\\
and other members of the OpenMP Interoperability Subcommittee}

\usepackage{comment}
\usepackage{needspace}
\usepackage[colorlinks=true,citecolor=blue]{hyper ref}
\usepackage{url}
\usepackage{xcolor}

% correct bad hyphenation here
\usepackage{makeidx}  % allows for indexgeneration
\usepackage{fancyvrb}
\usepackage{amsmath, amssymb}
\usepackage{epsfig}
\usepackage{setspace}\usepackage{multirow}
\usepackage{subfigure}
\usepackage{verbatim}
\usepackage{listings}
\usepackage{wrapfig}
%\usepackage{hyperref}
%\usepackage{xltxtra}
\newcommand{\REM}[1]{}

\newcommand{\descheader}[1]{{\needspace{3\baselineskip}\vspace{1em}\noindent \fbox{#1}}}


\begin{document}                                                
\maketitle
\section{Introduction}
Parallel and large-scale applications are typically developed using multiple parallel programming interfaces in 
a hybrid model, e.g. MPI+OpenMP, and using one or mulitple prebuilt scientific and/or platform-specific libraries such as MKL.
Each of these programming models and libraries has their own runtime to handle scheduling of work units and management of computational
and data movement tasks. There have been challening issues for using these models in one application, including 
compatibility issues for compiling and linking, oversubscription of resources at runtime, and the naming conflicts that 
programmers have to create workaround wrappers to deal with.

This report propose solutions to the interoperability and composability challenges faced by OpenMP programming interface, includling those
between multiple OpenMP implementations and/or multiple OpenMP runtime instances of the same implementation, OpenMP 
with native threads (pthreads and Windows Native threads), OpenMP with other threading languages and library such 
as C++11, TBB and Cilkplus, and OpenMP with inter-node programming models such as MPI, PGAS implemenation, etc. We 
think the similar challenges exist in other threading based libraries and language implementations, and believe
the solutions we provided in this technical report will work for them too.  

Interoperability and composability are closely related, while the interoperability sounds to improve the interactions between multiple models
while composability is meant to improve the modular use of OpenMP with itself and other models. One is from the aspect of system while 
the other is more concerned with software engineering. Both should be considered when developing solutions. 

For parallel programming languages and libraries, most implementations rely on system native threading (pthread or Windows Native threads) 
mechanisms to acquires system resources. Each implementation of the same or different programming models has their own mechanism for scheduling
user-level tasks and operations, which is the core part of a runtime system. 
The interoperability challenges are then concerned with how much we want two or more 
runtime instances (for the same or different high-level programming interfaces) to interact with other other for computational resource sharing
and data movement. Thus solutions to these challenges are more in the scope of runtime and implementation, than in the level of programming
interfaces and compiler transformations. 

\section{Topics and Interests}

\subsection{User threads}
\textbf{Definition}
A user thread is a thread that is not created by OpenMP implementation. A user thread 
could become an OpenMP initial thread. 

The most common example of user threads are POSIX Threads, 
usually referred to as Pthreads with implementation available on most Unix-like 
POSIX-compliant systems. There are also implementations of other thread libraries, 
for example, Windows Native threads and language based threading support such as 
Java threads or others (e.g. qthreads). 

Figure~\ref{fig:pthread-omp} shows an example of having three user 
threads (two PThreads and one thread of the main program) in an OpenMP program. 
The two three threads execute in parallel after the two PThreads are created. 
Each thread calls a function that will enter into
OpenMP threading parallelism. So they all become OpenMP initial threads. How the 
user threads (PThreads in this example) interact with the OpenMP threading mechanisms
in the runtime is up to the implementation. They may share the same OpenMP runtime
instance or each has its own OpenMP runtime instance. 

\begin{figure}[t]
\centering
  \fbox{
 % \lstset{basicstyle=\ttfamily\scriptsize,language=c}
  \lstset{basicstyle=\ttfamily\scriptsize,language=c,numbers=left, %,frame=single,
  deletekeywords={int,if,else,while},
  morekeywords={pragma,omp,target,device,map,
  tofrom,to,from,alloc,parallel,shared,reduction,data,collapse,
  private,dist_iteration,match_range,halo,exchange},
  numbersep=12pt,numberstyle=\color{red}}
  \lstinputlisting{pthread-omp.c}

}
\caption{Three user threads (two Pthreads and one main thread) with OpenMP}
  \label{fig:pthread-omp}
\end{figure}

\subsection{Impacts and Discussions}
A user thread in a program adds additional level(s) in the overall ``threading''
hierarchy of a program. Those additional levels could be on top of OpenMP threading 
mechanism when a user thread becomes an OpenMP initial thread that creates OpenMP thread
parallelism, or beneath the OpenMP threading mechanism when an OpenMP thread spawns 
a user thread, or the the mix of both. In the example from Figure~\ref{fig:pthread-omp}, 
one can view this in a two-level threading parallelism: the top level user thread 
parallelism and the bottom level OpenMP threading parallelism.

These additional levels of threading increase the complexity of a program, both for 
users in the aspect of reasoning the parallel and synchronization behavior of a program, 
and also for the implementation and runtime system in terms of resource management and 
interactions. Adding to the complexity is the facts that a user thread may be created 
through a call to a library function whose paralelism (OpenMP) behavior is not known to 
the callee. Typical issues for example: 
\begin{itemize}
\item Does each user thread use the same OpenMP runtime libraries or not? 
	If not using the same library, how to handle symbol name 
	conflicts of two more different OpenMP runtime libraries. 
\item For user threads that use the same OpenMP runtime library, does the user threads each create its own runtime instance or they share one?
\item For user threads each of which has its own runtime instance (from the same or 
	different runtime library), how to coordinate the resource management among those
	runtime instances to address such issues as oversubscriptions and the affinity
	between user threads?
\end{itemize}

It is important to note that approaches to address those issues are very implementation
dependent, requiring protocol and agreement in the runtime behavior and/or interfaces 
of different OpenMP implementations. It may not be realistic to solve some of the issue
from the language standard, and should be left to users to deal with them. In this
aspect, we still hope this report could provide userful information and practices 
for users. 



\subsection{Oversubscription}
Oversubscription happens when resources are claimed and held than what is needed.
A program may request more OpenMP threads than the total amount of hardware
threads available when entering a parallel region, which causes excessive competition 
among OpenMP threads for hardware threads and increases runtime overhead. 
When program execution enters into sequential stage after exiting a parallel region, 
those native threads that support the OpenMP threads in the parallel region may still 
alive in the background consuming CPU cycles. This 
will make those hardware threads unavailable to others. 
Oversubscription impact the performance of an applications and the system, 
but should not introduce correctness issue to a program. 

%A typical OpenMP runtime creates a pool of native threads who will execute OpenMP
% parallel regions and/or tasks. 

The two scenarios we mentioned above are the two kinds of oversubscription we should try to avoid:
{\bf 1) Active oversubscription}: Claiming or requesting more threads than 
what are available by the system.
{\bf 2) Passive oversubscription}: Thread resources are not released 
after parallel execution. Holding hardware threads after parallel execution may not 
always hurt the performance overall, e.g. it will improve the start-up performance of the 
upcoming parallel region. In this category, we are concerning those situations that 
actually impact the performance negatively.



\subsection{Interaction between contention groups}

\subsection{Affinity with user threads}
\subsubsection{Coherence group (domains)}

\subsection{Common Interface}
\subsubsection{Application Binary Interfaces(ABI)}

\subsection{Interoperability with node-level programming model, e.g. MPI}

\begin{comment}
\subsection{Use Cases}
We have identified several use cases of OpenMP interoperating with itself and other parallel programming models. 
\subsubsection{OpenMP with native threads}
\subsubsection{OpenMP with TBB/MKL}
\subsubsection{OpenMP with parallel scientific library, such as MKL}
\subsubsection{Linking libraries and objects built with different OpenMP compilers}
\subsubsection{OpenMP with inter-node model, e.g. MPI}

\subsection{Issues and Posibble Solutions}
\subsubsection{Active Oversubscription}
\subsubsection{Passive Oversubscription}
\subsubsection{Interop across contention group}
\end{comment}

\section*{Acknowledgments}

The authors would like to acknowledge A, B, C and \ldots

\clearpage
\bibliographystyle{abbrv}
\bibliography{interop}

\appendix
\clearpage

\section{OpenMP 4.1 Formal specification}
The complete OMP\_DYNAMIC spec reads: 
``The OMP\_DYNAMIC environment variable controls dynamic adjustment of the number of threads
to use for executing parallel regions by setting the initial value of the dyn-var ICV. The value of
this environment variable must be true or false. If the environment variable is set to true, the
OpenMP implementation may adjust the number of threads to use for executing parallel
regions in order to optimize the use of system resources. If the environment variable is set to
false, the dynamic adjustment of the number of threads is disabled. The behavior of the program
is implementation defined if the value of OMP\_DYNAMIC is neither true nor false.''


The complete OMP\_WAIT\_POLICY spec reads: 
``The OMP\_WAIT\_POLICY environment variable provides a hint to an OpenMP
 implementation about the desired behavior of waiting threads by setting the wait-policy-var ICV. A
 compliant OpenMP implementation may or may not abide by the setting of the environment
 variable.
 The value of this environment variable takes the form:
 ACTIVE \textbar  PASSIVE. 
 The ACTIVE value specifies that waiting threads should mostly be active, consuming processor
 cycles, while waiting. An OpenMP implementation may, for example, make waiting threads spin.
 The PASSIVE value specifies that waiting threads should mostly be passive, not consuming
 processor cycles, while waiting. For example, an OpenMP implementation may make waiting
 threads yield the processor to other threads or go to sleep.
 The details of the ACTIVE and PASSIVE behaviors are implementation defined.''

 
The complete OMP\_THREAD\_LIMIT spec reads:
``The OMP\_THREAD\_LIMIT environment variable sets the maximum number of OpenMP threads to use in a contention group by setting the thread-limit-var ICV.
The value of this environment variable must be a positive integer. The behavior of the program is implementation defined if the requested value of OMP\_THREAD\_LIMIT is greater than the number of threads an implementation can support, or if the value is not a positive integer.''


%\clearpage

\end{document}

